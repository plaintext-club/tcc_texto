\documentclass[../../layout.tex]{subfiles}

\begin{document}
\begin{resumo}
\hspace*{3em}As tecnologias de comunicação transformaram a vida das pessoas permitindo que elas se conectem umas às outras através de dispositivos eletrônicos de fácil porte e utilização como celulares e computadores.
Uma vez provada a eficiência e conveniência desses dispositivos, seu uso se expandiu de forma a integrar comunicação com sistemas de diferentes finalidades localizados em mercados, shoppings e hospitais.
Por meio da internet esses dispositivos conectam-se entre si e trocam informações em uma fração de segundo.
Sendo assim, pesquisas nesta área de dispositivos conectados são de grande valia para facilitar o dia a dia das pessoas e negócios.
O uso destes dispositivos não se restringe ao simples acionamento de cargas e podem atingir até mesmo a análise de dados em massa (\emph{Big Data}) e a Inteligência Artificial.
Este modelo de dispositivos conectados é chamado de IoT (Internet das Coisas) e é a base do nosso projeto, que visa desenvolver um sistema que inclua facilidade para o usuário final ao utilizar dispositivos conectados entre si, com uma interface de controle amigável sem a dependência de serviços de terceiros e sem que o usuário tenha conhecimento profundo em programação ou eletrônica.
Os métodos utilizados para criação desta solução foram escolhidos com base na praticidade, eficiência e aplicabilidade das ferramentas e principalmente nos conhecimentos adquiridos ao longo do curso.
\vspace{\onelineskip}

\noindent
\textbf{Palavras-chave}: Internet das Coisas. Dispositivos conectados. Código Aberto.
\end{resumo}
\end{document}
