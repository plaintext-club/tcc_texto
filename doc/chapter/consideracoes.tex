\documentclass[../../layout.tex]{subfiles}

\begin{document}
\chapter{Considerações Finais}
A plataforma desenvolvida teve como objetivo trazer os recursos de uma aplicação \emph{IoT} que independe do nível técnico do usuário. Objetivamos alcançar pessoas com diferentes níveis de conhecimento técnico para assim difundir a tecnologia para a maior quantidade de indivíduos possível. Além disso, a plataforma possui baixo custo o que torna a tecnologia ainda mais palpável. Para atingir esse objetivo foi necessário desenvolver uma interface amigável e intuitiva.\par
A interface é composta por um painel com cartões que podem ser adicionados ou removidos o que traz versatilidade para a plataforma. Dessa forma, o usuário é capaz de configurar a plataforma de maneira dinâmica, fácil e intuitiva. Cada um desses cartões podem representar um GPIO do \emph{hardware}, a criação de um cartão e atribuição do GPIO é de responsabilidade total do usuário, reforçando a versatilidade da plataforma. Além da escolha do GPIO o usuário pode configurar a direção dos pinos (saída ou entrada do hardware), nome do cartão e ainda o ícone de representação do cartão, como um botão, uma apresentação gráfica ou apenas o valor decimal lido pelo GPIO.\par
Durante o desenvolvimento da plataforma o maior obstáculo foi realizar os primeiros programas com a linguagem Elixir por ser uma linguagem de paradigma funcional diferente do método tradicional de programar. Entretanto, durante o desenvolvimento percebemos que foi uma ótima opção por ser uma linguagem de alto nível que não afeta a performance da aplicação e além de fornecer inúmeros recursos para acelerar o desenvolvimento de aplicações para sistemas embarcados.\par
Para a validação da plataforma realizamos testes de integração e funcional, visando mitigar possíveis falhas para o produto final. Para avaliar a interface foi necessário realizar o teste de usabilidade que foca na experiência do usuário portanto foi necessário a interação de usuários (com baixo nível técnico) com a plataforma a fim de comprovar a facilidade e a interatividade da interface. Todos os usuários conseguiram criar uma simples aplicação, sem grandes dificuldades, portanto a interface se mostrou intuitiva e amigável.\par
A plataforma desenvolvida atingiu o seu propósito inicial, porém para que se torne um produto final ainda são necessárias melhorias e desenvolvimento de novos recursos. Dentre esses podemos listar: a criação da autenticação do usuário com senha e \emph{login} para restringir o acesso da plataforma apenas ao proprietário dos dispositivos; criar uma carcaça que identifica as conexões dos sensores de forma clara para evitar conexões incorretas e diminuir a exposição do \emph{hardware}.\par

\end{document}
