\documentclass[../../layout.tex]{subfiles}

\begin{document}
\chapter{Introdução}
\hspace*{3em}Com o desenvolvimento elevado da tecnologia, a quantidade de dispositivos conectados à internet hoje em dia é bem maior que nas décadas anteriores. Estipula-se que obteve um aumento de cinco vezes em dez anos \acite{conectdevicesnum}. O uso desses dispositivos são variados e muitos deles estão dentro de nosso alcance, seja em casa ou serviços que utilizamos. Houveram ações e investimentos para que o desenvolvimento dessa tecnologia fosse impulsionada dentro do Brasil, especificamente para a área de IoT (\emph{internet of things}) voltada para telecomunicações \acite{iotinvest}. Nos dias atuais com tantos serviços utilizados dentro de casa, seja para acessar um site, ouvir música, assistir vídeo ou navegar no GPS, estamos utilizando um dispositivo conectado e por trás dele tem uma empresa que controla esse acesso e provém facilidade e informações para o usuário final, seja uma pessoa com o intuito de automatizar a casa ou deixá-la mais inteligente, ou até mesmo para controlar aplicações em indústrias. Muitas plataformas e serviços só foram capazes devido à esse avanço, nos proporcionando conforto e praticidade diariamente, porém, devido à isso, muitas empresas cobram por esses serviços.\par
Cada vez mais pessoas estão conectadas e inseridas na tecnologia, onde os conceitos básicos da tecnologia têm se tornado cada vez mais comum na sociedade, tornando possível  que pessoas com baixo nível de conhecimento técnico consigam compreender e desenvolver projetos simples. Nos últimos anos surgiram diversos projetos na área de \emph{IoT} por conta das facilidades e praticidades oferecidas por essa tecnologia, no entanto grandes obstáculos para desenvolver projetos são encontrados, como plataformas de preço elevado e que demandam alto nível de conhecimento técnico.\par
Os projetos em geral são considerados complexos pois envolvem diversas áreas da tecnologia, por este motivo são considerados como sistemas heterogêneos, envolvendo conhecimentos em programação de sistemas embarcados, análise de esquemas eletrônicos, redes de computadores, protocolos e arquitetura WEB.\par
Esses fundamentos são essenciais para o desenvolvimento de uma simples aplicação \emph{IoT}, portanto, para permitir que pessoas com baixo conhecimento técnico tenham acesso a um ambiente simples de desenvolvimento \emph{IoT}, é necessário condensar o desenvolvimento desses recursos em uma única plataforma que transcreve de forma intuitiva as configurações essenciais para o usuário desenvolver.\cite{IoTeveryone} \par
A proposta desse projeto é desenvolver uma plataforma genérica que permite que pessoas de baixo ou alto nível de conhecimento técnico possam realizar uma aplicação \emph{IoT} de forma fácil e intuitiva.

\section{Objetivo}
\hspace*{3em}Este projeto têm seu valor implementado tanto em software quanto hardware para centralizar um sistema intuitivo e prático com os seguintes objetivos:
\begin{enumerate}[label=\alph*)]
\itemsep0em
\item visualizar periféricos e dispositivos conectados ao sistema;
\item transmitir as informações dos dispositivos para o \emph{Raspberry host};
\item controlar periféricos e dispositivos através da interface web, mesmo fora de casa;
\item proporcionar ao usuário a flexibilidade de programar as funcionalidades desta plataforma de forma fácil, sem exigir conhecimentos de programação;
\item disponibilizar interface de baixo nível para maior liberdade de programação do usuário, incentivando o conhecimento e aprendizado.
\end{enumerate}


\end{document}
